\documentclass[a4paper,12pt]{article} 

\usepackage{graphicx}  % for inserting figure
\usepackage{amsmath}    % for maths equation
\usepackage{cite}       % for citation
\usepackage{natbib}     % for citation
\usepackage{float}    % for figure position
\usepackage{listings}   % for code
\usepackage{algorithm}  % for algorithm
\usepackage{algpseudocode}  % for algorithm

\usepackage{setspace}  % for line spacing
\usepackage{amssymb}    % for maths equation
\usepackage{caption}    % for inserting figure
\usepackage{subcaption} % for inserting figure
\usepackage{booktabs}   % for table
\usepackage{siunitx}    % for table
\usepackage{enumitem}   % for list
\usepackage{program}    % for program

\title{Latex basic}
\author{Atik Mouhtasim}
\date{\today}



\begin{document}
\maketitle




This is my 1st latex document. \\
here double black slash is used for  new line  \\ \\
Or we can use slash newline for new line \newline

We use slash special character to print: \! \% \$ \& \# \_   etc \\

Now lets make bold and italic text: \\
\textbf{To bold, we use slash textbf}\\
\bf we can also use bf to bold text\\
\textit{To make italic, we use slash textit}\\ 
\textbf{\textit{Both bold and italic}}\\
\underline{We use slash underline for undelining}\\


For, equation we use amsmath package\\

The equation of straight line is : $y=mx+c$ \\
(this is the 1st method, see code) \\
or, we can use equation environment:
\begin{equation}
    y=mx+c
\end{equation} 


\begin{equation}
    y=\int_{-\infty}^{\infty} (x^2 +2) dx
    \label{eq2}
\end{equation}



We use slash newpage for new page:
\newpage % Go to next page




Now to insert a figure we use graphicx package: \\
\begin{figure}[!hbt]
    \centering
    \includegraphics[width=100px]{kitty.png}
    \caption{Kitten image}
\end{figure}






For table we use tabular environment \\

\begin{table}[hbt!]
    \caption{} % title of Table
    \centering
    \begin{tabular}{|c|c|c|} % centered columns (4 columns)
        \hline
        col 1 & col 2 & col 3 \\ % name of the columns
        \hline % horizontal line
        5     & 6             \\ % value of the cells
        \hline
        7     & 8             \\
        \hline
    \end{tabular}
\end{table}



for parting the document we use section command: \\
\section{Introduction}
First section
\subsection{Type A}
1st subsection
\subsection{Type B}
2nd subsection
\subsubsection{Type B(a)}
1st subsubsection
\section{Conclusion}
Second section\\



Lets do some quotation: \\
\begin{quotation}
    \begin{spacing}{1.2}
        Quoted statements are also printed with both side
        aligned, but in a narrowed width.
        \begin{flushright}
            {\it - Name}
        \end{flushright}
    \end{spacing}
\end{quotation}





We can use paragraph and subparagraph this way:
\paragraph{(1) Title:}
    There are certain key issues to attract
    investors, which need to be addressed.

    \subparagraph{(1A) Title:}
        There are certain key issues to attract
        investors, which need to be addressed.
\par    % paragraph break
We use "slash par" for paragraph break \\ \\

\begin{center}
    Writing language: \hspace{5mm} English. \\
\end{center} 
Marks: 100 \hfill Time: 3 Hours.\\ \\






Now lets make a list:
\begin{enumerate}
    \item point 1
    \item point 2
    \item point 3
\end{enumerate}


We have dedicated package for algorithm: \\
\begin{algorithm}[H]
    \caption{: Insertion Sort}
    \begin{algorithmic}[1]
    \For{$j \gets 2$ to $Arr.lenght$}
    \State $key = Arr[j]$
    \State $i = j- 1$
        \While{$i>0$ and $Arr[i]>key$}
                \State $Arr[i+1]=Arr[i]$
                \State $i=i-1$
        \EndWhile
        \State $Arr[i+1]=key$
    \EndFor
    \end{algorithmic}
\end{algorithm}


Code for insertion sort: \\
\begin{verbatim}
    #include<stdio.h>
    #include<math.h>
    int main()
    {
    int n, a[101];
    int i, max;
    printf("Number of points = ");
    ...
    return(0);
    }
\end{verbatim}


\end{document}