\documentclass[12pt]{article}

\usepackage{amsmath}   % for maths equation
\usepackage{amssymb}


\begin{document}

\large

 

% eq 1
\begin{equation}
    y=mx+c
    \label{eq1}
\end{equation}


% for refering to the 1st equation
The equation (\ref{eq1}) is for a straight line. \\



\begin{equation*}
    \mbox{text}\quad x=x^\mathrm{low}+yd \enspace . %mbox for message  , \enspace for space
\end{equation*}


% eq 2
\begin{equation}
    y=mx^2+\sqrt{x}
    \label{eq2}
\end{equation}

% eq 3
\begin{equation}
    y=\sum_{0}^{N} n^2
    \label{eq3}
\end{equation}


% eq 4
\begin{equation}
    y=\sum_{0}x_{n}^2
    \label{eq4}
\end{equation}

% eq 5
\begin{equation}
    y=\int xdx
    \label{eq5}
\end{equation}


% eq 6
\begin{equation}
    y=\int_{0}^{\infty} xdx
    % \nonumber           % for not showing the equation number
\end{equation}


% eq 7
\begin{equation}
    y=\int_{-\infty}^{\infty} (x^2 +2) dx
    \label{eq7}
\end{equation}


% eq 8
\begin{equation}
    y=\frac{x-a}{x-b}
    \label{eq7}
\end{equation}

% eq 9
\begin{equation}
    a < b < c
    \label{eq8}
\end{equation}

% eq 10
\begin{equation}
    a \leq b \geq c
    \label{eq9}
\end{equation}



% eq 10
\begin{equation}
    f_x(x)=\left\{ \begin{array}{lll}
        0, & \mbox{if $x<0$}\\
        \frac{x-a}{a-b}, & \mbox{if $a \leq x < b$}\\
        1, & \mbox{if $x \geq b$}
    \end{array} \right.
\end{equation}

% eq 11
\begin{equation}
    \dot{x}, \ddot{x}, \dddot{x}, \ddddot{x} 
    \label{eq10}
\end{equation}

% eq 12
\begin{equation}
    \frac{\partial{y}}{\partial{x}}
\end{equation}

% eq 13
\begin{equation}
    \frac{dx}{dt} = \frac{d}{dt} (x^2 + 2x + 1)
\end{equation}


\begin{equation}
    \geq   
\end{equation}

\begin{equation}
    \ni
\end{equation}

\begin{equation}
    \propto
\end{equation}

\begin{equation}
    \gg
\end{equation}

\[x^2+y^2=r^2\] % for inline equation



\begin{flushleft}
    \bf Left aligning the equation
\end{flushleft}
\begin{flalign}   % for aligning multiple equations
    x^2+y^2=r^2 && \\
    x^2+y^2=r^2 &&
\end{flalign}


\begin{flalign*}  % for aligning multiple equations without numbering
    x^2+y^2=r^2 && \\
    x^2+y^2=r^2 &&
\end{flalign*}




\begin{equation*}
    \begin{array}{lll}
    \mbox{Minimize} & f(\mbox{\boldmath{$x$}}) &\\

    \mbox{Subject to}
    & g_i(\mbox{\boldmath{$x$}}) \leq 0 ; & i=1,\ldots,m\\
    & h_k(\mbox{\boldmath{$x$}}) = 0 ; & k=1,\ldots,p\\
    & x_j \geq 0 ; & j=1,\ldots,n
    \end{array}
\end{equation*}





\begin{multline}
    5x_1 + 2x_2 + 3x_3 -\\
    x_4 - 4x_5 + 5x_6 +\\
    7x_7 + 3x_8 - 6x_9 -\\
    2x_{10} - 5x_{11} = 7634
\end{multline}





\begin{equation}
    \begin{split}
    f(x) &= x^3 + 2x^2 - 5x + 10\\
    &= (2)^3 + 2(2)^2 - 5(2) + 10\\
    &= 16
    \end{split}
\end{equation}



\begin{equation}
    S = \frac{n}{2}(2a+\overline{n-1}d)
\end{equation}

\begin{equation}
    \begin{split}
    f(x,y)=h \biggl[ & \frac{1}{2}(x+y)+x^2+y^3\\
    & +\frac{1}{3}z^2\biggr]
    \end{split}
\end{equation}
% here underscore is used for subscript and cap is used for superscript. \\


\end{document}
