\documentclass[a4paper]{article}
\usepackage{tabularx}
\usepackage{rotating}
\usepackage{wrapfig}
\usepackage{lipsum}
\usepackage{longtable}


\begin{document}



\begin{table} % table environment
    \centering
    \caption{Obtained marks.}
    \label{tab-1}
    \begin{tabular}{|c|c|c|c|}
        \hline Name  & Math & Phy & Chem \\
        \hline Robin & 80   & 68  & 60   \\
        \hline Julie & 72   & 62  & 66   \\
        \hline
    \end{tabular}
\end{table}

Table \ref{tab-1} shows the Obtained marks. % for referencing


\begin{table}
    \centering
    \caption{Scored points.}
    \label{tab-2}
    \begin{tabularx}{0.8\linewidth}{|X|c| >{\raggedleft\arraybackslash}X|}
        % here, X is for flexible width, \raggedleft is for flushing right and \arraybackslash is for ignoring \\ 
        \hline {\bf Name} & {\bf Sex} & {\bf Points} \\
        \hline Milan      & M         & 1500         \\
        Julie             & F         & 1,325        \\
        Sekhar            & M         & 922          \\
        \hline
    \end{tabularx}
\end{table}

Table \ref{tab-2} shows the Scored marks \\ \\ % for referencing









We use slash hfill to make side by side tables.
\begin{table}[!hbt]  % !hbt means here, bottom, top
    % \centering
    \caption{Obtained marks. 2}
    \begin{tabular}{|l|c|c|c|}
        \hline Name   & \begin{sideways}Mathematics\end{sideways} & \begin{sideways}Physics\end{sideways} & \begin{sideways}Chemistry\end{sideways} \\
        \hline Robin  & 80                                        & 68                                    & 60                                      \\
        \hline Julie  & 72                                        & 62                                    & 66                                      \\
        \hline Robert & 75                                        & 70                                    & 71                                      \\
        \hline
    \end{tabular} \hfill
    % we can't use blank line here. Instead we will use % to comment out the blank line
    \begin{tabular}{|p{3.5cm}|m{1.5cm}|b{1.6cm}|} % p, m, b are for top, middle and bottom alignment(works on reverse)
        \hline This is a paragraph & A medium size entry & This is another long entry \\
        \hline
    \end{tabular}
\end{table}














{\bf Wrap table:} \\

\begin{wraptable}{r}{5cm}   % r means right, 5cm is the width
    \centering
    \label{wrap-table}
    \begin{tabular}{|l|c|c|c|}
        \hline Name & Math & Phy & Chem\\
        \hline Robin & 80 & 68 & 60\\
        \hline Julie & 72 & 62 & 66\\
        \hline Robert & 75 & 70 & 71\\
        \hline
    \end{tabular}
    \caption{Obtained marks - 3}
    \end{wraptable}
    If the size of a table is very small compared to the width of a page, the wraptable
environment, supported by the wrapfig package, can be used to wrap around the
table by texts. The wraptable environment needs two mandatory arguments, i.e.,
 where aside and asize are, respectively, the  and size of the table. The location can be specified by l (left side of the page)
or r (right side of the page), while the size is specified in units (e.g., 25mm, 1.0in,
or 0.3. The wraptable environment is similar with the table environment;
the only difference lies in creating the environment. \\
    \underline{Table \ref{wrap-table} has been shown as referencing.} \\ \\






\newpage



\underline{\bf Nested Table: }\\

\begin{table}[!hbt]
    \begin{tabularx}{\linewidth}{|l|X|c|}
        \hline semester & score & Total \\
        \hline First & {
            \begin{tabularx}{\linewidth}{X|c}
                English & 69 \\
                Science & 80 \\
                Drawing & 92
            \end{tabularx}
        } & 241 \\
        \hline
        \multicolumn{2}{|r|}{Total} & 241 \\    % 2 columns will be merged and the text will be right aligned
        \hline
    \end{tabularx}
\end{table}




\end{document}